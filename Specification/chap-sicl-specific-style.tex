\chapter{SICL-specific style guide}

\section{Commenting}

In most programs, comments introduce unnecessary redundancies that can
then easily get out of sync with the code.  This is less risky for an
implementation of a specification that is not likely to change.
Furthermore, we would like \sysname{} to be not only a high-quality
implementation, but we would like for its code to be very readable.
For that reason, we think it is preferable to write \sysname{} in a
``literate programming'' style, with significant comments explaining
the code. 

Accordingly, we prefer comments to consist of complete sentences,
starting with a capital letter, and ending with punctuation.

\section{Designators for symbol names}

Always use uninterned symbols (such as \texttt{\#:hello}) whenever a
string designator for a symbol name is called for.  In particular,
this is useful in \texttt{defpackage} and \texttt{in-package} forms.

Using the upper-case equivalent string makes the code break whenever
the reader is case-sensitive (and it looks strange that the designator
has a different case from the way symbol that it designates is then
used), and using keywords unnecessarily clutters the keyword package.

\section{Docstrings}

We believe that it is a bad idea for an implementation of a Lisp
system to have docstrings in the same place as the definition of the
language item that is documented, for several reasons.  First, to the
person reading the code, the docstring is most often noise, because it
is known from the standard what the language item is about.  Second,
it often looks ugly with multiple lines starting in column 1 of the
source file, and this fact often discourages the programmer from
providing good docstring.  Third, it makes internationalization
harder.

% in case you're interested in my opinion: docstrings are good.
% 0. Same place is easy to remember to fix
% 1. docstrings can be hidden via the editor
% 2. is a valid point - perhaps docstrings should be defined as ignoring 
%    whitespace and a ";" at the beginning of each line?
% 3. basically true, but shouldn't be an issue - English is the de-facto 
%    language. PLEASE PLEASE PLEASE don't make the localized-WinWord-Macro times 
%    come back, with a macro written in a German office installation being 
%    unusable in an English one and vice versa ...  ;)
% 4. even after 10 years doing CL I still often need the docstrings of the more 
%    verbose functions (OPEN etc.)

For this reason, we will provide language-specific files containing
all docstrings of Common Lisp in the form of calls to \texttt{(setf
documentation)}. 

We also recommend using \texttt{format} (at read time) so that the
format directive \texttt{\~{}@} can be used at the end of lines,
allowing the following line to be indented as the rest of the text.
That way, we avoid the ugliness of having subsequent lines start in
column 1.

\section{Naming and use of slots}

In order to make the code as safe as possible, we typically do not
want to export the name of a slot, whereas frequently, the reader or
the accessor of that slot should be exported.  This restriction
implies that a slot and its corresponding reader or accessor cannot
have the same name.  Several solutions exist to this problem.  The one
we are using for \sysname{} is to have slot names start with the
percent character (`\texttt{\%}').  Traditionally, a percent character
has been used to indicate some kind of danger, i.e. that the
programmer should be very careful before directly using such a name.
Client code that attempts to use such a slot would have to write
\texttt{package::\%name} which contains two indicators of danger,
namely the double colon package marker and the percent character.

Code should refer to slot names directly as little as possible.  Even
code that is private to a package should use an internal protocol in
the form of readers and writers, and such protocols should be
documented and exported whenever reasonable. 

\section{Standard functions}

Standard functions should always check the validity of their arguments
and of any other aspect of the environment.  If such a function fails
to accomplish its task, it should signal an appropriate condition.  

We would like error messages to be phrased in terms of the code that
was directly invoked by user code, as opposed to in terms of code that
was indirectly invoked by system code.  As an example, consider a
sequence function such as \texttt{substitute}.  If it is detected that
a dotted list has been passed to this function, it should not be
reported by \texttt{endp} or any other system function that was not
directly called by user code, but instead it should be reported by
\texttt{substitute} in terms of the sequence that was originally
passed as an argument.  On the other hand, if substitute invokes a
user-supplied test that fails, we would like the error message to be
reported in terms of that user-supplied code rather than by
\texttt{substitute}.  This is how we are currently imagining solving
this problem:

\begin{itemize}
\item Standard functions do not call any other standard functions
  directly, other than if it is known that no error will be signaled.
  When a call from a standard function $f$ to a standard function $g$
  might result in an error being signaled by $g$, that call is
  replaced by a call to a special version of the standard function,
  say $h$ that signals a more specific condition than is dictated by
  the Common Lisp \hs{}.
\item If acceptable in terms of performance, a standard function such
  as $f$ that calls other functions that may signal an error, handles
  such errors by signaling an error that is directly related to $f$. 
\item Error reporting is done in terms of the name and arguments to
  $f$. 
  
\end{itemize}

\section{Standard macros}

Standard macros must do extensive syntax analysis on their input so as
to avoid compilation errors that are phrased in terms of expanded
code.  

As with standard functions, standard macros that expand into other
system code that may signal an error should not use other standard
functions or other standard macros directly, but instead special
versions that signal more specific conditions.  The expanded code
should then contain a handler for such errors, which signals an error
in terms of the name and the arguments of the macro. 

\section{Compiler macros}

{\sysname} will make extensive use of compiler macros.  Compiler
macros are part of the standard, so this mechanism must be part of a
conforming compiler anyway.  In many cases, instead of encoding
special knowledge in the compiler itself, we can use compiler macros.
By doing it this way, we simplify the compiler, and we provide a set
of completely portable macros that any implementation can use. 

Compiler macros should be used whenever the exact shape of the call
site might be used to improve performance of the callee.  For
instance, when the callee uses keyword arguments, we can eliminate the
overhead of keyword-value parsing at runtime and instead call a
special version of the callee that does not have to do any such
parsing.%
\fixme{There is a suggestion that this creation could be automated by
  the compiler.  I don't know how doable that would be.}

Similarly, functions that take a \texttt{\&rest} argument can provide
special cases for different common sizes of the \texttt{\&rest}
argument.

We propose using compiler macros at least for the following
situations: 

\begin{itemize}
\item to convert calls to \texttt{list} and \texttt{list*} into nested
  calls to \texttt{cons};
\item to convert simple calls to some built-in functions that accept
  \texttt{:test} and \texttt{:key} keyword arguments (such as
  \texttt{find}, \texttt{member}, etc) into calls to
  special versions of these procedures with particularly simple
  functions for these keyword arguments (\texttt{identity},
  \texttt{car}, \texttt{eq}, etc);
\item to convert calls to some functions that accept optional
  arguments such as \texttt{last} and \texttt{butlast} into calls to
  special versions when the optional argument is not given.
\end{itemize}

Compiler macros should not be used in the place of inlining.
%(see section \ref{section-inlining}). doesn't exist?

\section{Conditions and restarts}

\sysname{} functions should signal conditions whenever this is
required by the Lisp standard (of course) and whenever it is
\emph{allowed} by the Lisp standard and reasonably efficient to do so.
If the standard allows for subclasses of indicated signals (I think
this is the case), then \sysname{} should generate as specific a
condition as possible, and the conditions should contain all available
information as possible in order reduce the required effort to
find out where the problem is located.

\sysname{} function should also provide restarts whenever this is
practical. 

\section{Condition reporting}

Condition reporting should be separate from the definition of the
condition itself.  Separating the two will make it easier to customize
condition reporting for different languages and for different
systems.  An integrated development environment might provide
different condition reporters from the normal ones, that in addition
to reporting a condition, displays the source-code location of the
problem. 

Every \sysname{} module will supply a set of default condition
reporters for all the specific conditions defined in that module.
Those condition reporters will use plain English text. 

\section{Internationalization}

We would like for {\sysname} to have the ability to report messages in
the local language if desired.  The way we would like to do that is to
have it report conditions according to a \texttt{language} object.  To
accomplish this, condition reporting trampolines to an
implementation-specific function \texttt{sicl:report-condition} which
takes the condition, a stream, and a language as arguments.

The value of the special variable \texttt{sicl:*language*} is passed
by the condition-reporting function to \texttt{sicl:report-condition}.

In other words, the default \texttt{:report} function for conditions is:

\begin{verbatim}
   (lambda (condition stream) 
     (sicl:report-condition condition stream sicl:*language*))
\end{verbatim}

Similarly, the Common Lisp function \texttt{documentation} should
trampoline to a function that uses the value of
\texttt{sicl:*language*} to determine which language to use to show
the documentation. 

\section{Package structure}

{\sysname} has a main package containing and exporting all Common Lisp
symbols.  It contains no other symbols.  A number of implementation
packages import the symbols from this package, and might define
internal symbols as well.  Implementation packages may export symbols
to be used by other implementation packages.

This package structure allows us to isolate implementation-dependent
symbols in different packages.  

\section{Assertions}

\section{Threading and thread safety}

Consider the use of locks to be free.  We predict that a technique
call ``speculative lock elision'' will soon be available in all main
processors.

% Ha! Quite the opposite. See Spectre, Meltdown, etc. Speculative executation is 
% a swear word currently ;)

