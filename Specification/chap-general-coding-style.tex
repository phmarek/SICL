\chapter{General \commonlisp{} style guide}

\section{Purpose of style restrictions}

The purpose of imposing a particular style is based on a few simple
facts that hold true for both natural languages and programming
languages:

\begin{itemize}
\item The set of all idiomatic phrases is a tiny subset of the set of
  all grammatical phrases.
\item The main purpose of these phrases is to serve as communication
  between people.
\end{itemize}

To illustrate the first fact, consider a natural language such as
English.  In English, we say ``tooth brush'', but ``dental floss''.
The words ``dental brush'' and ``tooth floss'' would be perfectly
grammatical, but they are not used.  A person trying to communicate
with other people must use the words that have been widely agreed
upon, even though some other words are perfectly legitimate.  It might
seem that such idiosyncrasies would be limited to languages with
multiple heritage such as English, but that is not the case.  In
French, we say ``brosse à dents'', ``pâte dentifrice'', and ``fil
dentaire''.  There is a large number of reasonable combinations, but
only one is used.

The same thing is true for programming languages.  The community has
collectively decided on a particular subset of all the grammatical
phrases, and a programmer who wishes to communicate with other
programmers should stick to that subset.

It should also be emphasized that the choice of idioms is different in
different languages.  An example from natural languages would be that
in English we say ``I wash my hands'', in French ``I wash myself the
hands'', and in Swedish we say ``I wash the hands [on myself]''.  Just
as it would be pointless trying to use an idiom from one language in a
translated version in a different language, it is as pointless to
translate idioms from one programming language to a different
programming language.

Finally, the choice of what phrases are idioms and what phrases are
not, is almost totally arbitrary, and based on coincidences of
history.  Therefore it is rarely productive to ask oneself why a
particular phrase is an idiom and a different one is not.  There is no
possible enlightening answer to such a question.


\section{Width of a line of code}

Horizontal space is a precious resource that should not be wasted.
The width of a line should preferably not exceed 80 characters.  This
limit used to be hard, because some printers or printer drivers would
truncate longer lines.  Since it is less common to print code these
days, the limit is now soft.  The purpose of keeping lines somewhat
short is so that it is possible on a reasonable monitor to display two
documents side by side.  One document is typically a \commonlisp{}
source file, and the other document is typically the buffer containing
interactions with the \commonlisp{} system.

The systematic use of long lines makes the practice of displaying two
documents side by side impossible, or at least impractical.  If a
single monitor is used, the programmer then has to flip back and forth
between the source code and the interaction loop.  When two monitors
are used, the effect is to waste half a monitor that could otherwise
be used for displaying documentation or something else.

\section{Commenting}

Use a single semicolon to introduce a comment that follows the code on
a line.  Use two semicolons for comments that are not at the top level
in a file and that should be aligned with the code that it comments
on.  Use three semicolons for top-level comments that concern some
top-level forms in a file, but not the entire file.  Use four
semicolons for comments that concern the entire file.

\section{Blank lines}

A single blank line is common in the following situations:

\begin{itemize}
\item Between two top-level forms.
\item Between a file-specific comment and the following top-level
  form.
\item Between a comment for several top-level forms and the first
  of those top-level forms.
\end{itemize}

A single blank line \emph{may} occur inside a top-level form to
indicate the separation of two blocks of code concerned with different
subjects, but it would be more common to put those two blocks of code
in separate functions.

There should never be any instance of two consecutive blank lines, and
the last line of the file should not be blank.

\section{\texttt{car}, \texttt{cdr}, \texttt{first}, etc are for
  \texttt{cons} cells}

The \commonlisp{} standard specifies that the function \texttt{car},
\texttt{cdr}, \texttt{first}, \texttt{second}, \texttt{rest}, etc
return \texttt{nil} when \texttt{nil} is passed as an argument.  This
fact should mostly be considered as a historical artifact and should
not be systematically exploited.  Take for instance the following
code:

\begin{verbatim}
(if (first x) ...)
\end{verbatim}

To the compiler, it means ``execute the false branch of the \texttt{if}
when either \texttt{x} is \texttt{nil}, or when \texttt{x} is a list
whose first element is \texttt{nil}''.

To the person reading the code, it means something different
altogether, namely ``\texttt{x} holds a non-empty list of Boolean
values, and the false branch of the \texttt{if} should be executed
when the first element of that list is
\emph{false}. See also \refSec{sec-coding-style-meanings-of-nil}.

\section{Different meanings of \texttt{nil}}
\label{sec-coding-style-meanings-of-nil}

Consider the following local variable bindings:

\begin{verbatim}
(let ((x '())
      (y nil)
      z)
  ...)
\end{verbatim}

To the compiler, the three are equivalent.  To a person reading the
code, they mean different things, however:

\begin{itemize}
\item The initialization of \texttt{x} means that \texttt{x} holds a
  \emph{list} that is initially empty.
\item The initialization of \texttt{y} means that \texttt{y} holds a
  Boolean value or a default value that may or may not change in the
  body of the \texttt{let} form.
\item The absence of initialization of \texttt{z} means that no
  initial value is given to \texttt{z}.  In the body of the
  \texttt{let} form, the variable \texttt{z} will be assigned to
  before it is used.
\end{itemize}

The following body of the \texttt{let} form corresponds to the
expectations of the person reading the code:

\begin{verbatim}
(let ((x '())
      (y nil)
      z)
  ...
  (push (f y) x)
  ...
  (unless y (setf y (g x)))
  ...
  (setf z (h x))
  ...)
\end{verbatim}

The following body of the \texttt{let} form violates the expectations
of the person reading the code:

\begin{verbatim}
(let ((x '())
      (y nil)
      z)
  ...
  (push (f y) z)     ; z is used before it is assigned.
  ...
  (unless x          ; x is treated as a Boolean.
    (setf y (g x)))
  ...
  (push (f x) y)     ; y is treated as a list.
  ...)
\end{verbatim}

\section{Tests in conditional expressions}

The \emph{test} of a conditional expression should be a (possibly
generalized) Boolean expression.  The following expressions correspond
to the expectations of the person reading the code:

\begin{verbatim}
(if visited-p ...)
(when (member ...) ...)
(cond ((plusp x) ...) ...)
\end{verbatim}

The following code violates the expectation:

\begin{verbatim}
(let ((item (find ...)))
  (when item ...))
\end{verbatim}

because \texttt{item} is not a (generalized) Boolean value.  It is an
item returned by \texttt{find}, though there is an \emph{out of band}
value (\texttt{nil}) indicating that no item was found by
\texttt{find}.  In this case, the corresponding code that corresponds
to the expectations would look like this:

\begin{verbatim}
(let ((item (find ...)))
  (unless (null item) ...))
\end{verbatim}

\section{General structure of recursive functions}

When possible, a recursive function should be structured like a
mathematical proof by induction.  By that we mean that the special
case should be handled \emph{first} so as to reassure the person
reading the code that this case can be handled correctly by the
function.


So for instance, assume we want to write a function that counts
the number of atoms in a tree, we should not write it like this:

\begin{verbatim}
(defun count-atoms (tree)
  (if (consp tree)
      (+ (count-atoms (car tree))
         (count-atoms (cdr tree)))
      1))
\end{verbatim}

but rather

\begin{verbatim}
(defun count-atoms (tree)
  (if (atom tree)
      1
      (+ (count-atoms (car tree))
         (count-atoms (cdr tree)))))
\end{verbatim}

Even when the base case does not return anything useful, it should be
handled first.  The following code violates the expectations:

\begin{verbatim}
(defun map-conses (function tree)
 (unless (atom node)
   (funcall function node)
   (traverse (car node))
   (traverse (cdr node))))
\end{verbatim}

and should be written like this instead:

\begin{verbatim}
(defun map-conses (function tree)
  (if (atom node)
      nil ; nothing to do
      (progn (funcall function node)
             (traverse (car node))
             (traverse (cdr node)))))
\end{verbatim}

though, admittedly, this example is a little too simple to illustrate
the importance of this rule.

\section{Using \texttt{car} and \texttt{cdr} vs. using \texttt{first}
  and \texttt{rest}}

While the two functions \texttt{car} and \texttt{first} have the exact
same definitions, as do \texttt{cdr} and \texttt{rest}, they send very
different messages to the person reading the code.

The functions \texttt{car}, \texttt{cdr}, etc., should be avoided when
the argument is to be considered as a \emph{list}, and should be
reserved for other uses of \texttt{cons} cells such as for
\emph{trees} or \emph{pairs} of values.

It follows that the two families of functions should never be mixed
for the same argument.
