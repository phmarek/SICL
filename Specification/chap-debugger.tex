\chapter{Debugger}
\label{chap-debugger}

Part of the reason for SICL is to have a system that provides
excellent debugging facilities for the programmer.  The kind of
debugger we plan to support is described in a separate repository.%
\footnote{See https://github.com/robert-strandh/Clordane}  In this
chapter, we describe only the support that \sysname{} contains in
order to make such a debugger possible.

When a function is compiled with a high value of the \texttt{debug}
quality, the execution of that function starts by loading a
\emph{flag} from the thread instance into a lexical variable.  During
compilation, that lexical variable is subject to register allocation
just like any other lexical variable.  This flag indicates whether the
current thread is being debugged.

The compiler inserts a call to a small subroutine before and after
every form to be evaluated.  The subroutine does not use the full
\commonlisp{} function-call protocol.  Instead, it is just a very fast
call that can be done with a \texttt{jsr} instruction (or equivalent)
on must architectures.

% If that function should be the same everywhere, a fixed register must be 
% chosen for the flag - so no "register allocation just like any other lexical 
% variable", right?

The subroutine starts by testing the flag to see whether the thread is
being debugged.  Under normal circumstances, the value of the flag is
\emph{false}, so the subroutine returns, and normal execution of the
function code is resumed.  If the flag should turn out to be
\texttt{true}, a second test is performed to see whether there is an
action to be taken for this particular value of the program counter.

This test is also done in two steps.  In the first step, the value of
the program counter is taken modulo some reasonably large value such
as 256, and a bit-table in the thread instance is queried to see
whether the corresponding entry is a 1.  If it is 0, again, the
subroutine again returns.  This first step will slow down every
debugged thread a little bit, but most of the time, the value will be
0, so again, normal function execution is resumed.

If the entry in the bit table turns out to be 1, then the final test
is made.  The program counter is checked against a hash table in the
thread instance to see whether some action must be taken.  If so, the
thread gives up control to the debugger.

%% For each possible breakpoint, the system must keep a description of
%% the lexical environment.  This includes mappings from variable names
%% to registers or stack locations, information about liveness of
%% registers and stack location, how a variable is stored in a location
%% (immediate value, pointer, with or without type tag, etc).

